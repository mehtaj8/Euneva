\documentclass[11pt, oneside]{article}   	% use "amsart" instead of "article" for AMSLaTeX format
\usepackage{geometry}                		% See geometry.pdf to learn the layout options. There are lots.
\geometry{letterpaper}                   		% ... or a4paper or a5paper or ... 
%\geometry{landscape}                		% Activate for rotated page geometry
%\usepackage[parfill]{parskip}    		% Activate to begin paragraphs with an empty line rather than an indent
\usepackage{graphicx}				% Use pdf, png, jpg, or eps§ with pdflatex; use eps in DVI mode
								% TeX will automatically convert eps --> pdf in pdflatex		
\usepackage{amssymb}

\usepackage{fancyhdr}
\pagestyle{fancy}

\lhead{Mehta, Jash
		- mehtaj8\\
		Sharma, Aditya
		- shara24\\
		Ren, Zackary
		- renx11}
\chead{SE 3XA3 PS-Rev0}
\rhead{\today}

\title{Problem Statement}
\author{}
%\date{}							% Activate to display a given date or no date
%\begin{header}

%\end{header}
\begin{document}
%\maketitle
\marginpar{}
\section*{Problem Statement Document}
\subsection*{1. What problem are you trying to solve?}

Students may be managing five, six or even seven courses in one semester in addition to any extracurricular commitments they may have. Each course has an array of quizzes and assignments that students can easily lose track of due to the lack of consolidated information. The way Avenue presents important information is more of a hindrance than a convenience and makes it harder for students to manage deadlines. This unnecessary stress can be detrimental to a student's academic performance and mental well-being.

\subsection*{2. Why is this an important problem?}

A recent study showed that 63\% of college students experienced overwhelming anxiety and 39\% have suffered from depression.$^1$ Additionally, over 40\% of college students take on a part time job. Balancing work, school and mental health is challenging, especially for younger individuals like students. Alleviating some of the stress from academics helps students avoid mental health issues and live a healthy lifestyle.\\

1. American College Health Association. Undergraduate Student Reference Group Fall 2018. https://www.acha.org/documents/ncha/NCHA-II\_Fall\_2018\_Undergraduate\_Reference \_Group\_Data\_Report.pdf 


\subsection*{3. What is the context of the problem you are solving?}
 
Our rendition of the Avenue-linked To-Do list is aimed directly towards students that attend McMaster University.Using a web-scraper, a list will be compiled with the weight of each assignment and the associated course, which would compliment the feature to automatically assign importance and/or undergo Pomodoro-based scheduling. This way, students who are stressed and “don’t know where to start” can use the product as a way to efficiently block out their study time.\\

Students spend an average of 7 hours on their phones throughout the day. This can be used as an advantage to create a software solution that will be a Mobile Application compatible with iOS or Android, making it accessible on the go. The application is geared towards “Educational and Organizational” type niches.

\newpage
\subsubsection*{3.1 Primary Stakeholders (Direct Benefit)}
\textbf{McMaster Students}
\begin{itemize}
\setlength\itemsep{1pt}
\item Gains
\begin{itemize}
\setlength\itemsep{1pt}
\item Ability to collate ALL assignments across courses 
\item Ability to plan in advance (maybe see in a calendar form)
\item Ability to involve a productivity method (ex. Pomodoro) with tasks
\end{itemize}

\item Pains
\begin{itemize}
\setlength\itemsep{1pt}
\item Key deadlines and amendments are lost across classes
\item Lack of centralized source of information
\end{itemize}
\end{itemize}

\textbf{Professors}
\begin{itemize}
\setlength\itemsep{1pt}
\item Gains
\begin{itemize}
\setlength\itemsep{1pt}
\item Ability to track student progress off of the To-Do list, good way to see if sufficient time has been allocated to each project
\end{itemize}

\item Pains
\begin{itemize}
\setlength\itemsep{1pt}
\item Most students don’t provide feedback, but complain about not having enough time
\end{itemize}
\end{itemize}

\subsubsection*{3.2 Secondary Stakeholders (Implicated in the Production of our Product)}

\textbf{McMaster University}
\begin{itemize}
\item Student Wellness Center
\begin{itemize}
\item Gains
\begin{itemize}
\item Ability to run anxiety awareness programs
\end{itemize}

\item Pains
\begin{itemize}
\item Decreased student engagement due to pandemic
\end{itemize}
\end{itemize}

\item Academic Success Center
\begin{itemize}
\item Gains
\begin{itemize}
\item Ability to teach time management \& productivity tips
\end{itemize}

\item Pains
\begin{itemize}
\item Ability to collate ALL assignments across courses 
\end{itemize}
\end{itemize}
\end{itemize}

\textbf{Brightspace (A2L Software)}
\begin{itemize}
\item Gains
\begin{itemize}
\item Compliments current product
\end{itemize}

\item Pains
\begin{itemize}
\item Software is organized so that rebuilding to funnel assignments to one source is complex (since each course is structured to be mutually exclusive)
\end{itemize}
\end{itemize}

\end{document}  
