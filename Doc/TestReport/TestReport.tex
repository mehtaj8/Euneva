\documentclass[12pt, titlepage]{article}

\usepackage{graphicx}
\graphicspath{ {./images/} }
\usepackage{booktabs}
\usepackage{tabularx}
\usepackage{float}
\usepackage{hyperref}
\hypersetup{
    colorlinks,
    citecolor=black,
    filecolor=black,
    linkcolor=red,
    urlcolor=blue
}
\usepackage[round]{natbib}
\usepackage[dvipsnames]{xcolor}
\usepackage[normalem]{ulem}

\title{SE 3XA3: Test Report\\Euneva}

\author{Team 9, Euneva
		\\ Mehta, Jash - mehtaj8
		\\ Sharma, Aditya - shara24
		\\ Ren, Zackary - renx11
}

\date{\today}
\begin{document}

\maketitle

\pagenumbering{roman}
\tableofcontents
\listoftables
\listoffigures

\begin{table}[bp]
\caption{\bf Revision History}
\begin{tabularx}{\textwidth}{p{3cm}p{2cm}X}
\toprule {\bf Date} & {\bf Version} & {\bf Notes}\\
\midrule
31/03/2021 & 1.0 & Initial version of Test Report (all members)\\
\bottomrule
\end{tabularx}
\end{table}

\newpage

\pagenumbering{arabic}

This document describes the test report generated after executing the Test Plan for Euneva. 

\section{Functional Requirements Evaluation}

\subsection{User Interface}

\begin{enumerate}
\item{test-UI1\\}

Type: Functional, Manual\\
Initial State: The app has been deployed\\
Input: N/A\\
Output: A boolean response\\
How test will be performed: An automated test will be written that checks if a button to log into Avenue is rendered\\
Result: PASS. The test confirmed a button enabling Avenue connection is rendered on the to-do list.

\item{test-UI2\\}

Type: Functional\\
Initial State: The app has been deployed\\
Input: N/A\\
Output: A boolean response\\
How test will be performed: An automated test will be written that checks if a button exists that adds an item to the list\\
Result: PASS. The test confirmed a button allowing the user to add an item to the list is rendered.

\item{test-UI3\\}

Type: Functional\\
Initial State: The app has been deployed\\
Input: N/A\\
Output: A boolean response\\
How test will be performed: An automated test will be written that checks if each item has a date assigned and displayed to it\\
Result: PASS. Each item added to the to-do has an associated date and is displayed properly.

\item{test-UI4\\}

Type: Functional\\
Initial State: The app has been deployed\\
Input: N/A\\
Output: A boolean response\\
How test will be performed: An automated test will be written that if the item name, due date, time and update button occurs when editing the item\\
Result: PASS. When adding a new item to the to-do list, the user has the option to enter the item name, due date, time and confirm the changes.

\item{test-UI5\\}

Type: Functional\\
Initial State: An item has been added\\
Input: User marks item as completed via click\\
Output: Item disappears from list\\
How test will be performed: A visual test will be performed to confirm that the item that has been clicked is no longer being rendered\\
Result: PASS. When an item is checked off as completed, it no longer appears on the to-do list.

\item{test-UI6}

Type: Functional\\
Initial State: An item has been added\\
Input: Random user input to an item’s name\\
Output: An update in the items information\\
How test will be performed: A visual test will be performed to confirm that when an item is edited, its changes are being reflected and rendered\\
Result: PASS. After performing an edit to an item, the changes are appropriately reflected on the to-do list. 

\end{enumerate}

\subsection{Scraping Functionality (SF)}

\begin{enumerate}
\item{test-SF1\\}

Type: Functional, Manual\\
Initial State: The app has been deployed\\
Input: User credentials for Avenue to Learn\\
Output: Console log, ‘logged on’ displayed\\
How test will be performed: A visual test will be performed to see if the correct value was displayed in the console.\\
Result: PASS. After entering the correct user credentials for Avenue to Learn, a successful log-in message was logged in the console. 

\item{test-SF2\\}

Type: Functional, Manual\\
Initial State: The app has been deployed\\
Input: N/A\\
Output: Console log, ‘Filtered Courses’ displayed\\
How test will be performed: A visual test will be performed to see if the correct value was displayed in the console.\\
Result: PASS. Upon logging in to Avenue and filtering courses, only the courses that are currently being taken were logged to the console.

\item{test-SF3\\}

Type: Functional, Manual\\
Initial State: The app has been deployed\\
Input: N/A\\
Output: Console log, ‘Assignment info collected’ displayed\\
How test will be performed: A visual test will be performed to see if the correct value was displayed in the console.\\
Result: PASS. The assignments of a particular course were requested. The output logged to the console was checked against the assignments seen on Avenue. 

\item{test-SF4\\}

Type: Functional, Manual\\
Initial State: The app has been deployed\\
Input: N/A\\
Output: Console log, ‘Quiz info collected’ displayed\\
How test will be performed: A visual test will be performed to see if the correct value was displayed in the console.\\
Result: PASS. The quizzes of a particular course were requested. The output logged to the console was checked against the quizzes seen on Avenue. 

\item{test-SF5\\}

Type: Functional, Manual\\
Initial State: The app has been deployed\\
Input: N/A\\
Output: Console log, ‘Filtered Information’ displayed\\
How test will be performed: A visual test will be performed to see if the correct value was displayed in the console.\\
Result: PASS. After filtering the current tasks of an individual, only tasks with future due dates and an incomplete status remained. 

\item{test-SF6\\}

Type: Functional, Manual\\
Initial State: The app has been deployed\\
Input: N/A\\
Output: The correct information is placed in the database and displayed on the UI\\
How test will be performed: A visual test will be performed to see if the correct information is displayed.\\
Result: PASS. After executing the Avenue scraping script, the database was correctly populated with the filtered information. 

\end{enumerate}

\section{Nonfunctional Requirements Evaluation}

\subsection{Look and Feel}

\begin{enumerate}
\item{test-LFA1\\}

Type: Manual\\
Initial State: The web app has been loaded\\
Input: N/A\\
Output: N/A\\
How test will be performed: A visual test will be performed to confirm that the product looks similar to Apple’s Reminders Application\\
Result: PASS. The to-do application was loaded and compared to Apple’s Reminders application and the visual appearance was deemed similar. 

\end{enumerate}

\subsection{Usability and Humanity}

\begin{enumerate}

\item{test-UHE1\\}

Type: Manual\\
Initial State: The web app has been loaded\\
Input: N/A\\
Output: N/A\\
How test will be performed: A interaction based assessment will be conducted by having an adult not involved in development use the product and provide their feedback in terms of learning to use the product and navigating the product\\
Result: PASS. A tester who had no prior experience with the to-do app was presented with the app. They managed to figure out the features intuitively and could use the app as intended. 

\end{enumerate}

\subsection{Performance}

\begin{enumerate}

\item{test-PSL1\\}

Type: Manual\\
Initial State: The backend of the application will be loaded\\
Input: Requests will be sent to each route\\
Output: The routes will retrieve and send back the data requested with the response time\\
How test will be performed: The backend server will be initialized and requests to all routes will be sent to it. The routes will return the data along with the response time. The response time should be less than the RESPONSE\_TIME.\\
Result: PASS. The contents of various courses was requested, the response time was logged to the console and confirmed that it remained lower than RESPONSE\_TIME. 

\item{test-PPA1\\}

Type: Automated\\
Initial State: The app has been deployed\\
Input: A task is added to the list\\
Output: The correct information about the task is saved\\
How test will be performed: An automated test will add different tasks to the list and check if the correct information is saved\\
Result: PASS. A variety of different tasks were added using the to-do list. The database was checked to confirm the item was successfully saved.

\item{test-PRA1\\}

Type: Manual\\
Initial State: The web app has been loaded\\
Input: User interacts with the application\\
Output: Application renders appropriate data on device\\
How test will be performed: A tester will confirm the application can be accessed on a device and upon interacting, the expected behaviour is observed. \\
Result: PASS. The test team accessed the app on different devices and after interacting with the app, confirmed that it is behaving as expected.

\item{test-PCR1\\}

Type: Manual\\
Initial State: The web app has been loaded\\
Input: User interacts with the application\\
Output: N/A\\
How test will be performed: The application will be used by NUMBER\_OF\_USERS at the same time and the performance will be evaluated.\\
Result: PASS. (!!!! HELP !!!!)

\end{enumerate}

\subsection{Operational and Environmental}

\begin{enumerate}
\item{test-OER1\\}

Type: Manual\\
Initial State: The web app has been loaded\\
Input: N/A\\
Output: N/A\\
How test will be performed: A tester will access the application on multiple different devices that can connect to the Internet\\
Result: PASS. Each tester could successfully access the application on different devices connected to the Internet.

\end{enumerate}

\subsection{Security}

\begin{enumerate}
\item{test-SR1\\}

Type: Manual\\
Initial State: The app has been loaded\\
Input: User enters their log-in credentials\\
Output: To-do list syncs with Avenue\\
How test will be performed: A tester will enter the log-in information of verified McMaster Students and random log-in information to confirm only McMaster students have access to the application.\\
Result: PASS. Entering the log-in credentials of a verified McMaster Student allowed access to the application. An incorrect log-in did not allow the individual to use the application. 

\end{enumerate}

\subsection{Cultural}

\begin{enumerate}
\item{test-CR1\\}

Type Manual\\
Initial State: The app has been loaded\\
Input: N/A\\
Output: N/A\\
How test will be performed: A tester will examine the application and ensure it is written in English and there are no inappropriate graphics or text.\\
Result: PASS. The application was inspected and all text in the application was in English. Additionally, no inappropriate or offensive text/graphics were found.

\end{enumerate}

\section{Comparison to Existing Implementation}

The existing implementation of the to-do list was built fine with modules exhibiting low coupling and high cohesion. The existing modules each had their own individual function and were not strongly related to one another. Each of the components of the user interface was clearly modularized, as such we did not have to modify any of the user interface components. We decided to introduce the ability to connect to Avenue to Learn and developed our own modules for this feature. To connect this feature to the application itself, we added two text fields and a button to the user interface allowing the user to enter their login credentials. Additionally, a task in the existing implementation did not have a date associated with it. We decided to add a date field as we felt a to-do list should keep track of the due date of a task. 

\section{Unit Testing}

Our team made the decision to forego unit testing due to the increased resources required for little to no gain. Since our app primarily focuses on the user interface and user experience, we opted to create a suite of manual tests to validate the various aspects of our app. 

\section{Changes Due to Testing}

\subsection{UI}
No changes were made due to testing
\subsection{Scraping Functionality}
No changes were made due to testing
\subsection{Look and Feel}
No changes were made due to testing
\subsection{Usability and Humanity}
No changes were made due to testing
\subsection{Performance}
No changes were made due to testing
\subsection{Operational and Environmental}
No changes were made due to testing
\subsection{Security}
No changes were made due to testing
\subsection{Cultural}
No changes were made due to testing

\section{Automated Testing}

A couple of the tests defined in the test plan were to be automated; however, during testing it was deemed infeasible. Given the time constraints, the team decided that the additional time investment required to set up the automated tests was not worth it. Instead we decided to perform manual tests and visually confirm our app was displaying the correct output. 

\section{Trace to Requirements}

\begin{table}[H]
\caption{Requirements Traceability Matrix}
\begin{center}
\begin{tabular}{ |m{14em}|m{6em}|} 
 \hline
 Requirement ID & Test ID(s) \\ 
 \hline
 FR1 & UI1 \\
 \hline
 FR2 & UI2\\ 
 \hline
 FR3 & UI3\\
 \hline 
 FR4 & UI4\\
 \hline
 FR5, FR7 & UI5\\
 \hline
 FR6 & UI6\\
 \hline
 FR8 & SF1\\
 \hline
 FR9 &SF2\\
 \hline
 FR10 & SF3\\
 \hline
 FR11 &SF4\\
 \hline
 FR12 & SF5\\
 \hline
 FR13 & SF6\\
 \hline
 LFA1, LFS1 & LFA1\\
 \hline
 UHE1, UHP1, UHA1 & UHE1\\
 \hline 
 PSL1 & PSL1\\
 \hline
 PPA1 & PPA1\\
 \hline
 PRA1, PRA2, PRF1 & PRA1\\
 \hline
 PCR1 & PCR1\\
 \hline
 EPE1, RIAS1, RR1, RR2 & OER1\\
 \hline
 AR1, AR2, IR1 & SR1\\
 \hline
 CPC1, CPC2 & CR1\\
 \hline
\end{tabular}
\end{center}
\label{traceabilityFunc}
\end{table}
A description of the requirements can be found in the \href{https://gitlab.cas.mcmaster.ca/renx11/3xa3-project-l02-group9/-/blob/master/Doc/SRS/SRS-Rev1.pdf}{SRS}.

\section{Trace to Modules}

The latest Module Guide can be found \href{https://gitlab.cas.mcmaster.ca/renx11/3xa3-project-l02-group9/-/blob/master/Doc/DesignDoc/MG/MG-Rev0.pdf}{here}.

\begin{table}[H]
\caption{Module Traceability Matrix}
\begin{center}
\begin{tabular}{ |m{8em}|m{12em}|} 
 \hline
 Test ID & Module(s) \\ 
 \hline
 UI1 & M1\\
 \hline
 UI2 & M1\\ 
 \hline
 UI3 & M1\\
 \hline 
 UI4 & M1\\
 \hline
 UI5 & M1\\
 \hline
 UI6 & M1\\
 \hline
 SF1 & M2\\
 \hline
 SF2 & M3\\
 \hline
 SF3 & M4\\
 \hline
 SF4 & M5\\
 \hline
 SF5 & M5\\
 \hline
 SF6 & M6\\
 \hline
 LFA1 & M1\\
 \hline
 UHE1 & M1\\
 \hline 
 PSL1 & M3, M6\\
 \hline
 PPA1 & M3, M6\\
 \hline
 PRA1 & M1, M2, M3, M6\\
 \hline
 PCR1 & M2\\
 \hline
 OER1 & M1, M2, M3, M4, M5, M6\\
 \hline
 SR1 & M1, M2, M3, M4, M5, M6\\
 \hline
 CR1 & M1\\
 \hline
\end{tabular}
\end{center}
\label{traceabilityFuncModule}
\end{table}

\section{Code Coverage Metrics}

The team decided against using code coverage due to the fact that all the tests being performed are all manual tests. Due to not having any automated test cases another way we confirm that the code coverage was met was through the examination of the manual tests and the traceability matrix between modules. This makes it clear that the important functionality has been tested and the team feels confident the software is reliable and consistent. 

\section{Appendix}

\subsection{Symbolic Parameters}

SP1. \textbf{RESPONSE\_TIME}: 2 seconds\\
SP2. \textbf{NUMBER\_OF\_USERS}: 50

\end{document}
