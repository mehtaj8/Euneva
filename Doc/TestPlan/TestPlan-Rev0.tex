\documentclass[12pt, titlepage]{article}

\usepackage{graphicx}
\graphicspath{ {./images/} }
\usepackage{booktabs}
\usepackage{tabularx}
\usepackage{float}
\usepackage{hyperref}
\hypersetup{
    colorlinks,
    citecolor=black,
    filecolor=black,
    linkcolor=red,
    urlcolor=blue
}
\usepackage[round]{natbib}

\title{SE 3XA3: Test Plan\\Euneva}

\author{Team 9, Euneva
		\\ Mehta, Jash - mehtaj8
		\\ Sharma, Aditya - shara24
		\\ Ren, Zackary - renx11
}

\date{\today}
\begin{document}

\maketitle

\pagenumbering{roman}
\tableofcontents
\listoftables
\listoffigures

\begin{table}[bp]
\caption{\bf Revision History}
\begin{tabularx}{\textwidth}{p{3cm}p{2cm}X}
\toprule {\bf Date} & {\bf Version} & {\bf Notes}\\
\midrule
04/03/2021 & 1.0 & Initial version of Test Plan (all members)\\
\bottomrule
\end{tabularx}
\end{table}

\newpage

\pagenumbering{arabic}

\section{General Information}

\subsection{Purpose}

The purpose of this document is to provide a plan for testing, validation, and verification procedures for the Avenue linked To-Do list software being implemented. The program will need to interpret user input and present accurate output. These components will be tested to build confidence that the software project will function as intended.

\subsection{Scope}

This document outlines the tests that will be done for testing the functional and non-functional requirements and proof of concept. The scope of testing will primarily cover the accuracy of the output and logic.

\subsection{Acronyms, Abbreviations, and Symbols}

\begin{table}[H]
\caption{Table of Abbreviations}
\begin{center}
\begin{tabular}{ |m{10em}|m{20em}|} 
 \hline
 Abbreviation & Definition \\ 
 \hline
 UI & User Interface \\
 \hline
 JSON & JavaScript Object Notation, a readable file format to transmit data objects\\ 
 \hline
 JS & JavaScript\\
 \hline 
 CSS & Cascading Style Sheets\\
 \hline
 TS & TypeScript\\
 \hline
\end{tabular}
\end{center}
\label{abbrev}
\end{table}

\begin{table}[H]
\caption{Table of Definitions}
\begin{center}
\begin{tabular}{ |m{10em}|m{20em}|} 
 \hline
 Term & Definition \\ 
 \hline
 Structural Testing & Type of testing carried out to test the structure of the code \\
 \hline
 Functional Testing & Type of software testing that validates the software system based on the functional requirements\\ 
 \hline
 Dynamic Testing & Type of testing that validates the behaviour and performance of the software system\\
 \hline 
 Static Testing & Type of testing where the actual program or application is not used\\
 \hline
 Manual Testing & Type of software testing that involves the Testers to annually execute the tests\\
 \hline
 Automated Testing & Type of software testing that involves the use of special testing software that controls the execution of tests and compares the results to the expected\\
 \hline
 Unit Testing & Type of software testing where the individual units/components of the software are tested\\
 \hline
 System Testing & Type of testing where a complete and integrated software is tested from the internal structure\\
 \hline
 Integration Testing & Type of software testing where individual components are tested in groups\\
 \hline
 React & An Open-Source application framework\\
 \hline
 Group-9 & Project Team\\
 \hline
\end{tabular}
\end{center}
\label{defs}
\end{table}

\subsection{Overview of Document}

This document will walk through the test plan for Euneva, this includes a plan for how the testing will be conducted, a system test description which goes over the tests for the functional and non-functional requirements, a plan for testing proof of concept, and a unit testing plan.

\section{Plan}

\subsection{Software Description}

This software will allow students from McMaster University to access Avenue to Learn assignments and quizzes for each class in one convenient location. The Front-End To-Do list application is made using the React web application framework, whilst the Back-End is implemented in Node. Python is used for the web-scraping done on Avenue to Learn.

\subsection{Test Team}

The test team will consist of Group-9 members involved in the development of the project. The test team members are, Jash Mehta, Aditya Sharma, Zackary Ren.

\subsection{Automated Testing Approach}

The overwhelming majority of our project’s functionality consists of user interaction. Therefore we determined that it would be costly to create a suit of unit tests of all base logic. Instead, our testing technique will be a manual test suite that ensures the correct responses to the user’s input. If each test passes it is implied that the underlying logical functionality of our program corresponds properly.

\subsection{Testing Tools}

The testing for this project will be done with PyTest, which is a testing suite framework for Python projects, which deals with unit testing and code coverage. As mentioned earlier, the Python component of our project is used for web-scraping; however, there are some functions that carry out tasks like data cleaning and data processing. These functions will be tested via PyTest.

\subsection{Testing Schedule}

See \href{https://gitlab.cas.mcmaster.ca/renx11/3xa3-project-l02-group9/-/blob/master/ProjectSchedule/EuenvaGantt.pdf}{Gantt Chart} for Testing Schedule.

\section{System Test Description}

\subsection{Tests for Functional Requirements}


\subsubsection{UI}

\begin{enumerate}
\item{test-UI1\\}

Type: Functional, Automated\\
Initial State: The app has been deployed\\
Input: N/A\\
Output: A boolean response\\
How test will be performed: An automated test will be written that checks if a button to log into Avenue is rendered\\
Mapping: FR1

\item{test-UI2\\}

Type: Functional\\
Initial State: The app has been deployed\\
Input: N/A\\
Output: A boolean response\\
How test will be performed: An automated test will be written that checks if a button exists that adds an item to the list\\
Mapping: FR2

\item{test-UI3\\}

Type: Functional\\
Initial State: The app has been deployed\\
Input: N/A\\
Output: A boolean response\\
How test will be performed: An automated test will be written that checks if each item has a date assigned and displayed to it\\
Mapping: FR3

\item{test-UI4\\}

Type: Functional\\
Initial State: The app has been deployed\\
Input: N/A\\
Output: A boolean response\\
How test will be performed: An automated test will be written that if the item name, due date, time and update button occurs when editing the item\\
Mapping: FR6

\item{test-UI5\\}

Type: Functional\\
Initial State: An item has been added\\
Input: User marks item as completed via click\\
Output: Item disappears from list\\
How test will be performed: A visual test will be performed to confirm that the item that has been clicked is no longer being rendered\\
Mapping: FR8, 10

\item{test-UI6\\}

Type: Functional\\
Initial State: An item has been added\\
Input: Random user input to an item’s name\\
Output: An update in the items information\\
How test will be performed: A visual test will be performed to confirm that when an item is edited, its changes are being reflected and rendered\\
Mapping: FR9

\end{enumerate}

\subsection{Tests for Non-Functional Requirements}

\subsubsection{Look and Feel Requirements}

\begin{enumerate}
\item {test-LFA1\\}

Type: Functional\\
Initial State: The web app has been loaded\\
Input: N/A\\
Output: N/A\\
How test will be performed: A visual test will be performed to confirm that the product looks similar to Apple’s Reminders Application\\
Mapping: LFA1, LFS1

\end{enumerate}

\subsubsection{Usability and Humanity Requirements}

\begin{enumerate}
\item{test-UHE1\\}

Type: Manual\\
Initial State: The web app has been loaded\\
Input: N/A\\
Output: N/A\\
How test will be performed: A interaction based assessment will be conducted by having an adult not involved in development use the product and provide their feedback in terms of learning to use the product and navigating the product\\
Mapping: UHE1, UHP1, UHA1
\end{enumerate}

\subsubsection{Performance Requirements}

\begin{enumerate}
\item{test-PSL1\\}

Type: Automated, Functional\\
Initial State: The backend of the application will be loaded\\
Input: Requests will be sent to each route\\
Output: The routes will retrieve and send back the data requested with the response time\\
How test will be performed: The backend server will be initialized and requests to all routes will be sent to it. The routes will return the data along with the response time. The response time should be less than the RESPONSE\_TIME\\
Mapping: PSL1

\item{test-PPA1\\}

Type: Automated\\
Initial State: The app has been deployed\\
Input: A task is added to the list\\
Output: The correct information about the task is saved\\
How test will be performed: An automated test will add different tasks to the list and check if the correct information is saved\\
Mapping: PPA1\\

\item{test-PRA1\\}

Type: Manual\\
Initial State: The web app has been loaded\\
Input: User interacts with the application\\
Output: Application renders appropriate data on device\\
How test will be performed: A tester will confirm the application can be accessed on a device and upon interacting, the expected behaviour is observed\\
Mapping: PRA1, PRA2, PRF1\\

\item{test-PCR1\\}

Type: Manual\\
Initial State: The web app has been loaded\\
Input: User interacts with the application\\
Output: N/A\\
How test will be performed: The application will be used by NUMBER\_OF\_USERS at the same time and the performance will be evaluated\\
Mapping: PCR1\\
\end{enumerate}

\subsubsection{Operational and Environmental Requirements}

\begin{enumerate}
\item{test-OER1\\}

Type: Manual\\
Initial State: The web app has been loaded\\
Input: N/A\\
Output: N/A\\
How test will be performed: A tester will access the application on multiple different devices that can connect to the Internet\\
Mapping: EPE1, RIAS1, RR1, RR2\\
\end{enumerate}

\subsubsection{Security Requirements}

\begin{enumerate}
\item{test-SR1\\}

Type: Manual\\
Initial State: The app has been loaded\\
Input: User enters their log-in credentials\\
Output: To-do list syncs with Avenue\\
How test will be performed: A tester will enter the log-in information of verified McMaster Students and random log-in information to confirm only McMaster students have access to the application\\
Mapping: AR1, AR2, IR1, PR1, PR2\\
\end{enumerate}

\subsubsection{Cultural Requirements}

\begin{enumerate}
\item{test-CR1\\}

Type: Manual\\
Initial State: The app has been loaded\\
Input: N/A\\
Output: N/A\\
How test will be performed: A tester will examine the application and ensure it is written in English and there are no inappropriate graphics or text\\
Mapping: CPC1, CPC2\\
\end{enumerate}

\subsection{Traceability Between Test Cases and Requirements}

The following table maps test cases to their respective requirements.

\begin{table}[H]
\caption{Requirements Traceability Matrix}
\begin{center}
\begin{tabular}{ |m{7em}|m{20em}|m{6em}|} 
 \hline
 Requirement ID & Description of Fit Criterion & Test ID(s) \\ 
 \hline
 FR1 & A button is rendered allowing the user to log-in to Avenue to Learn & UI1 \\
 \hline
 FR2 & A button allows the user to add a task to the to-do list & UI2\\ 
 \hline
 FR3 & The applications displays relevant information pertaining to a certain date & UI3\\
 \hline 
 FR6 & Display confirmation message when a task is added & UI4\\
 \hline
 FR8, FR10 & Task is removed from to-do list upon completion or deletion & UI5\\
 \hline
 FR9 & The user can edit a task & UI6\\
 \hline
 LFA1, LFS1 & The application will have an intuitive user flow and simplistic UI & LFA1\\
 \hline
 UHE1, UHP1, UHA1 & The user can navigate throughout the application with ease & UHE1\\
 \hline 
 PSL1 & The application responds to user actions after RESPONSE\_TIME & PSL1\\
 \hline
 PPA1 & The information of a task accurately reflects information on Avenue & PPA1\\
 \hline
 PRA1, PRA2, PRF1 & The application can be accessed on a device connected to the Internet & PRA1\\
 \hline
 PCR1 & The application can support NUMBER\_OF\_USERS at a time & PCR1\\
 \hline
 EPE1, RIAS1, RR1, RR2 & The application can be accessed on different types of devices connected to the Internet & OER1\\
 \hline
 AR1, AR2, IR1 & Only McMaster students can access the application & SR1\\
 \hline
 CPC1, CPC2 & The application displays text in English, without offensive graphics or text. & CR1\\
 \hline
\end{tabular}
\end{center}
\label{traceability}
\end{table}

\section{Tests for Proof of Concept}

\subsection{Application Testing}

\begin{enumerate}
\item{test-POC1\\}

Type: Functional, Dynamic, Manual\\
Initial state: The web app has been loaded\\
Input: The user interacts with the application\\
Output: The application displays appropriate response when the respective action is performed\\
How the test will be performed: The user will interact with different aspects of the application and the expected response will be visually confirmed
\end{enumerate}

\section{Unit Testing Plan}

\subsection{Unit Testing of Internal Functions}

Internal functions of the program will be methods that will have return values or execute a specific task. Unit tests for internal functions will involve having various inputs for the methods, and comparing the methods output to the expected output. Unit tests will include various inputs for methods consisting of normal inputs, boundary inputs, and inputs that will generate exceptions and errors for the program. The unit tests will display test cases that have passed, and test cases that have failed with appropriate reasoning behind the failure. Coverage methods, such as code coverage and branch coverage, will be used to determine how much of the program has been covered by the unit testing.

\subsection{Unit Testing of Output Files}

The application does not produce an output file. Therefore, there is no requirement to perform testing on an output file.

\section{Appendix}

\subsection{Symbolic Parameters}

SP1. \textbf{RESPONSE\_TIME}: 2 seconds\\
SP2. \textbf{NUMBER\_OF\_USERS}: 50

\subsection{Usability Survey Questions}

We want to question users on their experience with our application to gauge whether or not it accomplishes our goal of alleviating some stress from academics. \\

\begin{enumerate}
\item{Did you have any difficulties navigating throughout the application?}
\item{Did the user onboarding help with navigation around the application?}
\item{Did the use of the application reduce academic related stress?}
\item{Did it take long to get familiar with the application? If so, what felt unnatural?}
\item{Were there any features you wish were added?}
\end{enumerate}


\end{document}
